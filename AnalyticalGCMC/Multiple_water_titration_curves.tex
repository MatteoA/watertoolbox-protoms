\documentclass[10pt,a4paper]{article}
\usepackage[utf8]{inputenc}
\usepackage{parskip}
\usepackage{geometry} % to change the page dimensions
\geometry{a4paper} % or letterpaper (US) or a5paper or....
\geometry{margin=1in} % for example, change the margins to 2 inches all round
\usepackage{amsmath}
\usepackage{amsfonts}
\usepackage{amssymb}
\author{Gregory Ross}
\title{Analytic expressions titration curves for multisite binding  }

\begin{document}
\maketitle
\begin{abstract}

\section{Statistical mechanics}
This note outlines analytic expressions for the average number of molecules in a system derived from the grand canonical partition function. These expressions are used to validate the free energies calculated with the grand canonical integration (GCI) equation derived in the JACS manuscript. 
\end{abstract}

The grand canonical partition function, denoted $Z_G = Z_G(\mu,V,T)$, is a weighted sum of canonical partition functions, denoted $Z = Z(N,V,T)$:

\begin{align}
Z_G(\mu,V,T) &= \sum_{N=0}^\infty \exp(N\mu\beta) \, Z(N,V,T) \\
&= \sum_{N=0}^\infty \exp(N\mu\beta - F(N)\beta),
\end{align}

where $F(N,V,T)\beta = -\ln\big(Z(N,V,T)\big)$ has been inserted in the second line, and all unnamed terms have their standard thermodynamic meaning. Constant $V$ and $T$ will be assumed throughout, so their dependence will be henceforth remain implicit.

We are interested in how the average number of particles, denoted $\langle N \rangle$, varies with the applied chemical potential as this is integrated over in the GCI equation to give the ideal gas insertion free energies.  

From statistical mechanics (see, for instance, McQuarrie), 
\begin{align}
\langle N \rangle(\mu) &= -kT\frac{\partial\ln(Z_G(\mu))}{\partial\mu} \notag\\
&= \frac{\sum_{N=0}^\infty N\exp(N\mu\beta - F(N)\beta)}{\sum_{N=0}^\infty \exp(N\mu\beta - F(N)\beta)},
\label{eq:aveN}
\end{align}

which is nothing more than the ensemble average of $N$ at a fixed $\mu$.

The idea is to relate the above expression for $\langle N \rangle(\mu)$ to the GCI equation, which gives and expression for $F_\text{trans}$, defined as the free energy to transfer water molcules from ideal gas to the system of interest:

\begin{align}
F_\text{trans} = F(N) - F_\text{ideal}(N),
\end{align}

where $F_\text{ideal}(N)$ is the free energy for $N$ molecules of ideal gas. The GCI equation is given by

\begin{align}
 \beta(F_\text{trans}(N_f) - F_\text{trans}(N_i))  =  \langle N_f \rangle B_f -  \langle N_i\rangle B_i + \ln \left(\frac{\langle N_i \rangle !}{ \langle N_f\rangle !}\right) - \int_{B_i}^{B_f}  \langle N(B) \rangle d B, 
 \label{eq:gci}
\end{align}
where $F_\text{trans}(N_f) - F_\text{trans}(N_i)$ is the change in the transfer free energy to change the average number of particles from $\langle N_i \rangle$ to  $\langle N_f \rangle$, and $B$ is the Adams value. The Adams value is related to $\mu$ via

\begin{align}
B = \mu\beta + \ln\frac{V}{\Lambda^3}.
\end{align}

As the analytic expression for $\langle N \rangle(\mu)$ (equation \ref{eq:aveN}) depends on the free energies $F(N)$, these can be compared to the free energies calculated by GCI (equation \ref{eq:gci}), which integrates over $\langle N \rangle(B)$.

So that $\langle N \rangle(\mu)$ (equation \ref{eq:aveN}) can be related to GCI (equation \ref{eq:gci}), we need to express it terms of $B$. First, we require the explicit expression for $F_\text{ideal}(N)$:

\begin{align}
F_\text{ideal}(N)\beta = -\ln\left[\frac{1}{N!}\left(\frac{V}{\Lambda^3}\right)^N\right].
\end{align}

With this relation, the exponents in equation \ref{eq:aveN} can be expanded to

\begin{align*}
N\mu\beta - F(N)\beta  &= -N\mu\beta -  F_\text{trans}(N)\beta - F_\text{ideal}(N)\beta \\
&= N(\mu\beta + \ln\frac{V}{\Lambda^3}) - F_\text{trans}(N)\beta  - \ln(N!) \\
&= NB - F_\text{trans}(N)\beta  - \ln(N!)
\end{align*}

This allows us to rewrite equation \ref{eq:aveN} as 

\begin{align}
\langle N \rangle(B) = \frac{\sum_{N=0}^\infty N\exp(NB - F_\text{trans}(N)\beta  - \ln(N!))}{\sum_{N=0}^\infty \exp(NB - F_\text{trans}(N)\beta  - \ln(N!))}.
\label{eq:aveN2}
\end{align}

\section{Multisite binding}
In this section, expressions for $\langle N \rangle(B)$ (equation \ref{eq:aveN2}) will be explicitly written for systems that have capacities for up to three particles. All free energies will be defined in reference to $F(0)$, so that we set $F(0)=0$.

\subsection{Single site}
Dy definition, a single site system has a capacity for only one particle and adding any more particles is massively unfavourable. We represent this by setting all $F(N)=+\infty$ for all $N>1$. Putting this information into equation \ref{eq:aveN2} yields

\begin{align*}
\langle N \rangle(B) &= \frac{\exp(B - F_\text{trans}(1)\beta)}{1 + \exp(B - F_\text{trans}(1)\beta)} \\
&= \frac{1}{1 + \exp(F_\text{trans}(1)\beta - B)},
\end{align*}

which is the same as the logistic equation in the JACS manuscript. There, it was explicitly proven that $F_\text{trans}(1)$ is equal to the free energy as calculated using GCI.

\subsection{Two sites}
In this case, $F(N) = +\infty$ for all $N>2$, giving us

\begin{align*}
\langle N \rangle(B) = \frac{\exp(B - F_\text{trans}(1)\beta) + 2\exp(2B - F_\text{trans}(2)\beta - \ln(2))}{1 + \exp(B - F_\text{trans}(1)\beta) + \exp(2B - F_\text{trans}(2)\beta - \ln(2))}
\end{align*}

\subsection{Three sites}
Here, $F(N) = +\infty$ for all $N>3$, which gives us
\begin{align*}
\langle N \rangle(B) = \frac{\exp(B - F_\text{trans}(1)\beta) + 2\exp(2B - F_\text{trans}(2)\beta - \ln(2)) + 3\exp(3B - F_\text{trans}(3)\beta - \ln(6))}{1 + \exp(B - F_\text{trans}(1)\beta) + \exp(2B - F_\text{trans}(2)\beta - \ln(2)) + \exp(3B - F_\text{trans}(3)\beta - \ln(6))}
\end{align*}

And so on for more sites.

\subsection{Notebook}
The Jupyter notebook accompanying this write-up  numerically verifies the free energies calculated with GCI compared to the free energies contained in the expressions for $\langle N \rangle(B)$. The numerical agreement is excellent, and indicates that the GCI equation is correct.

\end{document}